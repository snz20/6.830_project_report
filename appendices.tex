
\section*{Appendix A: Installing Madlib with PostgreSQL}


These instructions have been tested on Ubuntu 12.04 x64. It won't work with Ubuntu 12.04 x32 (MADlib does not have support for Ubuntu 32-bit). Also, MADlib doesn't work with GCC 4.7.*, so since Ubuntu 12.10 ships with GCC 4.7.2 it might be an issue.

~~\\
\# install postgres packages \\
sudo apt-get -y install postgresql-9.1 libpq-dev postgresql-server-dev-9.1 postgresql-plpython-9.1

\# Download madlib from http://madlib.net and copy tar file to server \\
wget https://github.com/madlib/madlib/zipball/v0.5.0
unzip v0.5.0
cd madlib-madlib-5fabd88 

\# Build Madlib \\
sudo apt-get -y install cmake
sudo apt-get -y install m4 gcc-4.6 g++-4.6 g++
./configure
cd build/
make
make doc
make install

\# To connect to the database: \\
sudo su - postgres
psql

\# To add a password to a role: (note: might be more convenient to use a 1-letter password when coding for MADlib as we'll keep having to enter the password each time we reinstall it) \\
postgres=\# alter role postgres with password 'postgres';

\# To register Madlib with the postgres database \\
/usr/local/madlib/bin/madpack -p postgres -c \$USER@\$HOST/postgres install

\# To test your installation you can run the install check procedure: \\
/usr/local/madlib/bin/madpack -p postgres -c \$USER@\$HOST/\$DATABASE install-check


\section*{Appendix B: Creating a module in Madlib}

Create a new function testpy. We are going to create this section in any module called 'testmod'.

\begin{itemize}
  \item Add new module folder: ./src/ports/postgres/modules/testmod/
  \item Create in this folder the following files:

\begin{itemize}
  \item \_\_init\_\_.py\_in  
  \item testmod.py\_in
  \item testmod.sql\_in
\end{itemize}

\item Modify ./src/config/Modules.yml and add a new line "	- name: testmod"
  \item recompile MADlib: with any user:

\begin{itemize}
  \item ./configure
  \item make install      \# (no need for make clean) (in the MADlib build folder)
\end{itemize}
  \item re-register MADlib into Postgresql (with postgres user):
  \begin{itemize}
  \item  /usr/local/madlib/bin/madpack -p postgres -c \$USER@\$HOST/postgres reinstall
  \end{itemize}
 \end{itemize}
 


\section*{Appendix C: How to use Genetic Programming module in MADlib}

We are going to over an example that shows how to perform symbolic regression using the genetic programming module in MADlib. First we generate an artificial dataset using MATLAB that will contain 100,000 rows:

\begin{verbatim}
x1 = 1:0.0001:(10+5000);
x2 = 10:0.0001:(19+5000);
x3 = 5:0.0001:(14+5000);
a =[x1; x2; x3; x1.*(x2.^2+x3)]';
csvwrite('mock.csv', a(1:100000, :))
\end{verbatim}

{\raggedleft Then we import this dataset within PostgreSQL:}
\begin{verbatim}
CREATE TABLE mock (X1 real, X2 real, X3 real, Y1 real);
COPY mock FROM '/root/mock.csv' DELIMITERS ',' CSV;
\end{verbatim}

{\raggedleft Lastly we run the symbolic regression, with 100 individuals per population size, 20 generations, and to maximum size of the trees of 3. We take 3 attributes as input (x1, x2 and x3) and one attribute as output (y1).}
\begin{verbatim}
SELECT * from MADLIB.gp('mock', '{x1, x2, x3}', '{y1}', 100, 20, 3);
\end{verbatim}

{\raggedleft The above query produces the following output:}
\begin{verbatim}
SELECT * from MADLIB.gp('mock', '{x1, x2, x3}', '{y1}', 100, 20, 3);
           individual            |      fitness
---------------------------------+-------------------
 [mul, add, mul, x2, x2, x3, x1] | 0.000442927237356
 [mul, add, mul, x2, x2, x3, x1] | 0.000442927237356
 [mul, add, mul, x2, x2, x3, x1] | 0.000442927237356
 [mul, add, mul, x2, x2, x3, x1] | 0.000442927237356
 [mul, add, mul, x2, x2, x3, x1] | 0.000442927237356
 [mul, add, mul, x2, x2, x3, x1] | 0.000442927237356
 [mul, add, mul, x2, x2, x3, x1] | 0.000442927237356
 [mul, add, mul, x2, x2, x3, x1] | 0.000442927237356
 [mul, add, mul, x2, x2, x3, x1] | 0.000442927237356
(9 rows)
\end{verbatim}

Each row corresponds to one individual, i.e. one formula. In this example, all rows correspond to the formula $x_1*(x_2^2+x_3)$, which is what we wanted to find. The fitness reflects how accurate each individual is. The lowest the fitness, the most accurate the formula is.


\section*{Appendix D: How to use AdaBoost module in MADlib}
Currently our implementation supports only binary classification.

\subsection*{Input}
The {\itshape training data} as well as {\itshape test data} is expected to be of the following form:

\begin{verbatim}
TABLE tableName (
    id INTEGER, // 1 indexed
    attribute1 DOUBLE PRECISION,
    ...
    attributeN DOUBLE PRECISION,
    class INTEGER // should be either 1 or -1
)
\end{verbatim}

\subsection*{Usage}
Perform AdaBoost classification loading the whole dataset into memory: (This type of execution is pretty fast when the dataset is small and fits into memory)

\begin{verbatim}
postgres=# SELECT * from madlib.adaboost_train_and_classify (
               'trainingSet', 'testSet', 
               '{attribute1, ..., attributeN}', 
               'class', numberOfIterations, pValue
           );
\end{verbatim}

When the dataset cannot be fit into memory, the user can use either batched or row-by-row version of AdaBoost.
\vspace{\baselineskip}
{\raggedleft Perform row-by-row version of AdaBoost classification:}

\begin{verbatim}
postgres=# SELECT * from madlib.adaboostRow_train_and_classify (
               'trainingSet', 'testSet', 
               '{attribute1, ..., attributeN}', 
               'class', numberOfIterations, pValue
           );
\end{verbatim}

{\raggedleft Perform batched version of AdaBoost classification:}

\begin{verbatim}
postgres=# SELECT * from madlib.adaboostBatch_train_and_classify (
               'trainingSet', 'testSet', 
               '{attribute1, ..., attributeN}', 
               'class', numberOfBatches,
                numberOfIterations, pValue
           );
\end{verbatim}

\subsection*{Example}
This is an over-simplified example of the in-memory execution of AdaBoost classification. Batched and row-by-row versions are similar.
\vspace{\baselineskip}
{\raggedleft The training and test data:}

\begin{verbatim}
postgres=# SELECT * from trainingSet;
 id | attr1 | attr2 | attr3 | class 
----+-------+-------+-------+--------
  1 |    85 |    92 |    45 |     1 
  2 |    85 |    64 |    59 |    -1 
  3 |    86 |    54 |    33 |     1 
  4 |    91 |    78 |    34 |     1 
  5 |    87 |    70 |    12 |    -1 
  6 |    98 |    55 |    13 |     1 
(6 rows)

postgres=# SELECT * from testSet;
 id | attr1 | attr2 | attr3 | class 
----+-------+-------+-------+--------
  1 |    95 |    82 |    15 |    -1 
  2 |    86 |    54 |    54 |     1 
  3 |    88 |    64 |    43 |     1 
  4 |    81 |    70 |    46 |     1 
  5 |    97 |    77 |    10 |    -1 
  6 |    88 |    65 |    12 |    -1 
(6 rows)
\end{verbatim}

{\raggedleft Perform AdaBoost classification:}

\begin{verbatim}
postgres=# SELECT * FROM madlib.adaboost_train_and_classify (
               'trainingSet', 'testSet', 
               '{attr1, attr2, attr3}', 
               'class', 50, 0.05
           );
\end{verbatim}

{\raggedleft The above query produces the following output summary:}

\begin{verbatim}
      result_table_name      
-----------------------------
 adaboost_classifier_weights
 adaboost_classified_samples
 adaboost_confusion_matrix
 adaboost_AUC
(4 rows)
\end{verbatim}

{\raggedleft Check the contents of the above tables to get the classification results:}

\begin{verbatim}
postgres=# SELECT * FROM adaboost_classifier_weights;

 id |     weight      
----+-----------------
 0  | 0.0492970944955
 1  |  0.161430043833
 2  |  0.225351747846
 3  |  0.225351747846
 4  |  0.190980668725
 5  |  0.147588697256
(6 rows)

postgres=# SELECT * FROM adaboost_classified_samples;

 row_id |     score      | predicted_class 
--------+----------------+-----------------
 0      |  12.3265804125 |               1
 1      | -12.2369378612 |              -1
 2      |  11.9033472397 |               1
 3      |  11.9033472397 |               1
 4      |  10.7171405001 |               1
 5      | -12.0688375328 |              -1
(6 rows)

postgres=# SELECT * FROM adaboost_confusion_matrix;

 tp | fp | fn | tn 
----+----+----+----
  2 |  2 |  1 |  1
(1 row)

postgres=# SELECT * FROM adaboost_AUC;

      auc       
----------------
 0.444444444444
(1 row)

\end{verbatim}




