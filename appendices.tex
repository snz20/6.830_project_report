
\section*{Appendix A: Installing Madlib with PostgreSQL}
The following instructions have been tested on Ubuntu 12.04 x64. They will not work with Ubuntu 12.04 x32 (MADlib does not have support for Ubuntu 32-bit). Also, MADlib doesn't work with GCC 4.7.*. Since,  Ubuntu 12.10 ships with GCC 4.7.2 it might be an issue while installing MADlib on Ubuntu 12.10.

\vspace{\baselineskip}
{\raggedleft Install PostgreSQL packages:}
\begin{verbatim}
$ sudo apt-get -y install \
  postgresql-9.1 libpq-dev \
  postgresql-server-dev-9.1 \
  postgresql-plpython-9.1
\end{verbatim}

{\raggedleft Download MADlib from \url{http://madlib.net} and copy .tar file to server:}
\begin{verbatim}
$ wget https://github.com/madlib/madlib/zipball/v0.5.0
$ unzip v0.5.0
$ cd madlib-madlib-5fabd88
\end{verbatim}

{\raggedleft Build Madlib:}
\begin{verbatim}
$ sudo apt-get -y install cmake
$ sudo apt-get -y install m4 gcc-4.6 g++-4.6 g++
$ ./configure
$ cd build/
$ make
$ make doc
$ make install
\end{verbatim}

{\raggedleft Connect to the database:}
\begin{verbatim}
$ sudo su - postgres
$ psql
\end{verbatim}

{\raggedleft Add a password to a role:}
\begin{verbatim}
postgres=# ALTER ROLE postgres WITH PASSWORD 'postgres';
\end{verbatim}

{\raggedleft Register Madlib with the PostgreSQL database}
\begin{verbatim}
$ /usr/local/madlib/bin/madpack -p postgres -c \ 
  \$USER@\$HOST/postgres install
\end{verbatim}

{\raggedleft Test the installation by running install check procedure:}
\begin{verbatim}
$ /usr/local/madlib/bin/madpack -p postgres -c \
  \$USER@\$HOST/\$DATABASE install-check
\end{verbatim}

\section*{Appendix B: Creating a module in Madlib}

Create a new function testpy. We are going to create this section in any module called 'testmod'.

\begin{itemize}
  \item Add new module folder: ./src/ports/postgres/modules/testmod/
  \item Create in this folder the following files:

\begin{itemize}
  \item \_\_init\_\_.py\_in  
  \item testmod.py\_in
  \item testmod.sql\_in
\end{itemize}

\item Modify ./src/config/Modules.yml and add a new line "	- name: testmod"
  \item recompile MADlib: with any user:

\begin{itemize}
  \item ./configure
  \item make install      \# (no need for make clean) (in the MADlib build folder)
\end{itemize}
  \item re-register MADlib into Postgresql (with postgres user):
  \begin{itemize}
  \item  /usr/local/madlib/bin/madpack -p postgres -c \$USER@\$HOST/postgres reinstall
  \end{itemize}
 \end{itemize}
 


\section*{Appendix C: How to use Genetic Programming module in MADlib}
\subsection*{Input}
The input data is expected to be of the following form:
\begin{verbatim}
TABLE tableName (
    x1 DOUBLE PRECISION,
    ...
    xN DOUBLE PRECISION,
    y DOUBLE PRECISION
)
\end{verbatim}

\subsection*{Usage}
Perform Symbolic Regression using Genetic Programming:
\begin{verbatim}
postgres=# SELECT * FROM madlib.gp (
               'inputTableName', 
               '{x1, ..., xN}', '{y}', 
                numIndividuals, 
                numGenerations, 
                maxDepthOfTree
           );
\end{verbatim}

\subsection*{Example}
This is an example showing how to perform symbolic regression using the genetic programming module in MADlib. 

\vspace{\baselineskip}
{\raggedleft Generate an artificial dataset using MATLAB that contains 100,000 rows:}

\lstset{language=Matlab}
\lstset{tabsize=4}
\begin{lstlisting}
x1 = 1:0.0001:(10+5000);
x2 = 10:0.0001:(19+5000);
x3 = 5:0.0001:(14+5000);
a =[x1; x2; x3; x1.*(x2.^2+x3)];
csvwrite('mock.csv', a(1:100000, :))
\end{lstlisting}

{\raggedleft Import this dataset within PostgreSQL:}
\begin{verbatim}
postgres=# CREATE TABLE mock (X1 real, X2 real, X3 real, Y1 real);
postgres=# COPY mock FROM '/root/mock.csv' DELIMITERS ',' CSV;
\end{verbatim}

{\raggedleft Run the symbolic regression, with 100 individuals per population size, 20 generations, and to maximum size of the trees of 3. We take 3 attributes as input (x1, x2 and x3) and one attribute as output (y1):}

\begin{verbatim}
postgres=# SELECT * from madlib.gp('mock', '{x1, x2, x3}', '{y1}', 100, 20, 3);
\end{verbatim}

{\raggedleft The above query produces the following output:}
\begin{verbatim}
           individual            |      fitness
---------------------------------+-------------------
 [mul, add, mul, x2, x2, x3, x1] | 0.000442927237356
 [mul, add, mul, x2, x2, x3, x1] | 0.000442927237356
 [mul, add, mul, x2, x2, x3, x1] | 0.000442927237356
 [mul, add, mul, x2, x2, x3, x1] | 0.000442927237356
 [mul, add, mul, x2, x2, x3, x1] | 0.000442927237356
 [mul, add, mul, x2, x2, x3, x1] | 0.000442927237356
 [mul, add, mul, x2, x2, x3, x1] | 0.000442927237356
 [mul, add, mul, x2, x2, x3, x1] | 0.000442927237356
 [mul, add, mul, x2, x2, x3, x1] | 0.000442927237356
(9 rows)
\end{verbatim}

{\raggedleft Each row corresponds to one individual, i.e. one formula. In this example, all rows correspond to the formula $x_1*(x_2^2+x_3)$, which is what we wanted to find. The fitness reflects how accurate each individual is. The lowest the fitness, the most accurate the formula is.}


\section*{Appendix D: How to use AdaBoost module in MADlib}
\label{sec:adaapp}
Currently our implementation supports only binary classification.

\subsection*{Input}
The {\itshape training data} as well as {\itshape test data} is expected to be of the following form:

\begin{verbatim}
TABLE tableName (
    id INTEGER, // should be 1 indexed
    attribute1 DOUBLE PRECISION,
    ...
    attributeN DOUBLE PRECISION,
    class INTEGER // should be either 1 or -1
)
\end{verbatim}

\subsection*{Usage}
Perform AdaBoost classification loading the whole dataset into memory: (This type of execution is pretty fast when the dataset is small and fits into memory)

\begin{verbatim}
postgres=# SELECT * from madlib.adaboost_train_and_classify (
               'trainingSet', 'testSet', 
               '{attribute1, ..., attributeN}', 
               'class', numberOfIterations, pValue
           );
\end{verbatim}

When the dataset cannot be fit into memory, the user can use either batched or row-by-row version of AdaBoost.
\vspace{\baselineskip}
{\raggedleft Perform row-by-row version of AdaBoost classification:}

\begin{verbatim}
postgres=# SELECT * from madlib.adaboostRow_train_and_classify (
               'trainingSet', 'testSet', 
               '{attribute1, ..., attributeN}', 
               'class', numberOfIterations, pValue
           );
\end{verbatim}

{\raggedleft Perform batched version of AdaBoost classification:}

\begin{verbatim}
postgres=# SELECT * from madlib.adaboostBatch_train_and_classify (
               'trainingSet', 'testSet', 
               '{attribute1, ..., attributeN}', 
               'class', numberOfBatches,
                numberOfIterations, pValue
           );
\end{verbatim}

\subsection*{Example}
This is an over-simplified example of the in-memory execution of AdaBoost classification. Batched and row-by-row versions are similar.
\vspace{\baselineskip}
{\raggedleft The training and test data:}

\begin{verbatim}
postgres=# SELECT * from trainingSet;
 id | attr1 | attr2 | attr3 | class 
----+-------+-------+-------+--------
  1 |    85 |    92 |    45 |     1 
  2 |    85 |    64 |    59 |    -1 
  3 |    86 |    54 |    33 |     1 
  4 |    91 |    78 |    34 |     1 
  5 |    87 |    70 |    12 |    -1 
  6 |    98 |    55 |    13 |     1 
(6 rows)

postgres=# SELECT * from testSet;
 id | attr1 | attr2 | attr3 | class 
----+-------+-------+-------+--------
  1 |    95 |    82 |    15 |    -1 
  2 |    86 |    54 |    54 |     1 
  3 |    88 |    64 |    43 |     1 
  4 |    81 |    70 |    46 |     1 
  5 |    97 |    77 |    10 |    -1 
  6 |    88 |    65 |    12 |    -1 
(6 rows)
\end{verbatim}

{\raggedleft Perform AdaBoost classification:}

\begin{verbatim}
postgres=# SELECT * FROM madlib.adaboost_train_and_classify (
               'trainingSet', 'testSet', 
               '{attr1, attr2, attr3}', 
               'class', 50, 0.05
           );
\end{verbatim}

{\raggedleft The above query produces the following output summary:}

\begin{verbatim}
      result_table_name      
-----------------------------
 adaboost_classifier_weights
 adaboost_classified_samples
 adaboost_confusion_matrix
 adaboost_AUC
(4 rows)
\end{verbatim}

{\raggedleft Check the contents of the above tables to get the classification results:}

\begin{verbatim}
postgres=# SELECT * FROM adaboost_classifier_weights;

 id |     weight      
----+-----------------
 0  | 0.0492970944955
 1  |  0.161430043833
 2  |  0.225351747846
 3  |  0.225351747846
 4  |  0.190980668725
 5  |  0.147588697256
(6 rows)

postgres=# SELECT * FROM adaboost_classified_samples;

 row_id |     score      | predicted_class 
--------+----------------+-----------------
 0      |  12.3265804125 |               1
 1      | -12.2369378612 |              -1
 2      |  11.9033472397 |               1
 3      |  11.9033472397 |               1
 4      |  10.7171405001 |               1
 5      | -12.0688375328 |              -1
(6 rows)

postgres=# SELECT * FROM adaboost_confusion_matrix;

 tp | fp | fn | tn 
----+----+----+----
  2 |  2 |  1 |  1
(1 row)

postgres=# SELECT * FROM adaboost_AUC;

      auc       
----------------
 0.444444444444
(1 row)

\end{verbatim}




